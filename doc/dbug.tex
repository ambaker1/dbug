\documentclass{article}

% Input packages & formatting
\input{template/packages}
\input{template/formatting}
\renewcommand{\cleartooddpage}[1][]{\ignorespaces} % single side
\newcommand{\caret}{$^\wedge$}

% Other macros
\renewcommand{\^}[1]{\textsuperscript{#1}}
\renewcommand{\_}[1]{\textsubscript{#1}}

\title{\Huge Tcl Debugging Tools\\\small Version 0.1.0}
\author{Alex Baker\\\small\hyperlink{https://github.com/ambaker1/dbug}{https://github.com/ambaker1/dbug}}
\date{\small\today}
\begin{document}
\maketitle
\begin{abstract}
Since OpenSees is a script-based finite element analysis software, it can be difficult to debug a model or analysis when problems arise.
Typically, \textit{puts} statements are the extent of debugging a script written in Tcl, but this method can be cumbersome for more complex scripts.
Towards making OpenSees Tcl more user-friendly, the ``dbug'' package makes debugging easy, with tools such as the pause command.
\\\\
Requires the package wob, which can be downloaded at \hyperlink{https://github.com/ambaker1/wob}{https://github.com/ambaker1/wob}
\end{abstract}
\clearpage
\section{Pausing a Script} 
The \cmdlink{pause} command pauses a Tcl script, prints the file and line number, and enters command-line mode, allowing the user to query variables and insert code into an analysis. If the command entered while paused returns an error, the error message will be displayed and the script will remain paused. If the command entered is ``return'', the pause will be exited and the corresponding result and options will be passed to the caller. For example, a loop can be broken by entering \textit{return -code break} in pause mode. Pressing enter with no commands will simply continue the script.
\begin{syntax}
\command{pause}
\end{syntax}
\begin{example}{Pausing an analysis}
\begin{lstlisting}
pause
\end{lstlisting}
\tcblower
\begin{lstlisting}
PAUSED...
File: C:/User/Documents/MyFile.tcl
Line: 407
> 
\end{lstlisting}
\end{example}
Note: The pause command cannot be used in interactive mode (no need to pause when in command-line mode), but a pause can be used within a pause.
\clearpage
\section{Advanced Tcl Debugger}
The \cmdlink{debug} command parses a Tcl script, printing out intermediate steps of Tcl evaluation. Optionally, a recursive maximum depth to step into procedures can be specified. Commands are shown in the command window with a > prefix, and procedure depth is shown with asterisks. Each command is matched with a result line, thus showing the commands and their intermediate results. Note that this will slow down execution dramatically.
\begin{syntax}
\command{debug} \$script <\$depth>
\end{syntax}
\begin{args}
\$script & Tcl script to debug. \\
\$depth & Optional recursive depth to step into procedures (default 0)
\end{args}

\begin{example}{Debugging a procedure}
\begin{lstlisting}
proc add {a b} {
    return [expr {$a + $b}]
}
set a 5
set b 7
debug {
    add [expr {$a*2}] $b
} 1
\end{lstlisting}
\tcblower
\begin{lstlisting}
--------------------------------------------------------------------------------
COMMAND:
> add [expr {$a*2}] $b
--------------------------------------------------------------------------------
STEPS:
> expr {$a*2}
10
> add 10 7
****> expr {$a + $b}
****17
****> return 17
17
--------------------------------------------------------------------------------
\end{lstlisting}
\end{example}
\clearpage
\section{Verbose Command Evaluation}
The command \cmdlink{>} simply prints the fully substituted command, with prefix ``>'', and the results.
It is kind of the opposite of the comment symbol ``\#'', and is not intended to be used in interactive mode.
\begin{syntax}
\command{>} \$arg1 \$arg2 ...
\end{syntax}
\begin{args}
\$arg1 \$arg2 ... & Command arguments.
\end{args}
\begin{example}{Verbose evaluation}
\begin{lstlisting}
set a 5
set b 7
> expr $a + $b
\end{lstlisting}
\tcblower
\begin{lstlisting}
> expr 5 + 7
12
\end{lstlisting}
\end{example}
\clearpage
\section{Unit Testing}
The command \cmdlink{assert} can be used for unit testing of Tcl scripts. Throws error if the statement is false.
If the statement is true, it simply returns nothing and the script continues.
\begin{syntax}
\command{assert} \$value1 <\$op \$value2>
\end{syntax}
\begin{args}
\$value1 & Value to test. \\
\$op & Comparison operator. Default ``==''. \\
\$value2 & Comparison value. Default ``true''.
\end{args}
\begin{example}{Validation for unit testing}
\begin{lstlisting}
assert [string is double 5.0]; # Asserts that 5.0 is a number
assert [expr {2 + 2}] == 4; # Asserts that math works
\end{lstlisting}
\end{example}

\clearpage
\section{Viewing Variables} 
The \cmdlink{pvar} command is a short-hand function for printing the name and value of Tcl variables, including arrays.
\begin{syntax}
\command{pvar} \$name1 \$name2 ...
\end{syntax}
\begin{args}
\$name1 \$name2 ... & Name(s) of variables to print
\end{args}

\begin{example}{Printing variables to screen}
\begin{lstlisting}
set a 5
set b(1) 2
set b(2) 3
pvar a b
\end{lstlisting}
\tcblower
\begin{lstlisting}
a = 5
b(1) = 2
b(2) = 3
\end{lstlisting}
\end{example}


\subsection{Interactive Workspace Viewer} 
Alternatively, the command \cmdlink{viewVars} can be used to pause a Tcl script and opens up an interactive table of all variables in the current scope and their values. Variables cannot be edited in the variable viewer window, but their values can be selected and copied. This command in particular requires the packages ``Tk'' and ``Tktable''.
\begin{syntax}
\command{viewVars}
\end{syntax}

\end{document}
